%!TEX root=../document.tex

\section{Aufgabenstellung}
Suche dir mit einem/einer Partner/in ein Python Webframework aus und präsentiere deine Lösung!

\subsection{Grundanforderungen(70\%)}
Hier werden die zu erwerbenden Kompetenzen und deren Deskriptoren beschrieben. Diese werden von den unterweisenden Lehrkräften vorgestellt.

Dies kann natürlich auch durch eine Aufzählung erfolgen:

\begin{itemize}
	\item Installiere und konfiguriere eines der präsentierten Frameworks
	\item Überlege dir einen sinnvollen Anwendungsfall für das Framework und erstelle eine passende Datenbank dazu
	\item Erstelle eine simple Seite, welche Datensätze aus einer Datenbank anzeigt
	\item Protokolliere deine Vorgehensweise, aufgetretene Probleme etc.
\end{itemize}


\subsection{Erweiterungen(30\%)}
\begin{itemize}
	\item Verwende ein ansprechendes Design
	\item Das Bearbeiten von Datensätzen
	\item Das Löschen von Datensätzen
	\item Das Erstellen von neuen Datensätzen
\end{itemize}


\subsection{Abgabe}
\textbf{Protokoll} inkl. Arbeitsaufwand, Screenshots und Beschreibungen
\clearpage
